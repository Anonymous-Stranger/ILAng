% Section on how to use ILA to model interrupts.

\section*{Interrupt}
For general purpose processors, the instruction-level abstraction (ILA) can not 
only capture the operations defined by its instruction-set architecture, but also
the behavior of interrupts.
As defined in the ILA definition, an instruction can be fetched from both the 
input port and the architectural states.

%
Take a processor with interrupt service routine (ISR) stored in a certain range 
of instruction memory, such as 8051 micro-controller, as an example.
When the interrupt signal is raised, the processor stores the current program 
counter into the stack and updates the program counter to the entry point of
the ISR.
This can be modeled with the following strategies:
%
\begin{enumerate}
\item Include the interrupt signal as part of the instruction.
\item Evaluate the decode functions of normal instructions to \textit{false} if 
    any interrupt signal is raised.
\item Model the interrupt handling with a new instruction whose decode 
    function is evaluated to \textit{true} when the interrupt signal is raised.
\item The next state function of the new interrupt instruction updates the 
    program counter to the ISR entry point.
\end{enumerate}

%
Note that, in this example, instructions in the ISR are also normal 
instructions. 
%
The below shows part of an ILA template of the above example in modeling 
interrupts.
\lstinputlisting{code/interruptExample.py}
