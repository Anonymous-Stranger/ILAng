%%%%%%%%%%%%%%%%%%%%%%%%%%%%%%%%%%%%%%%%%%%%%%%%%%%%%%%%%%%%%%%%%%%%%%%%%%%%%%%%
%2345678901234567890123456789012345678901234567890123456789012345678901234567890
%        1         2         3         4         5         6         7         8

\documentclass[letterpaper, 11 pt]{article}  % Comment this line out
                                                          % if you need a4paper
%\documentclass[a4paper, 10pt, conference]{ieeeconf}      % Use this line for a4
                                                          % paper
\usepackage{url}
\usepackage{graphicx}
\usepackage{amsmath}
\usepackage{amssymb}
\usepackage{amsfonts}
\usepackage{proof}
\usepackage{tikz}
\usepackage[margin=1.2in]{geometry}


\usepackage{color}
\definecolor{light-gray}{gray}{0.95}
\usepackage{listings}
\lstset{ %
language=Python,                % choose the language of the code
basicstyle=\footnotesize,       % the size of the fonts that are used for the code
numbers=left,                   % where to put the line-numbers
numberstyle=\footnotesize,      % the size of the fonts that are used for the line-numbers
stepnumber=1,                   % the step between two line-numbers. If it is 1 each line will be numbered
numbersep=5pt,                  % how far the line-numbers are from the code
backgroundcolor=\color{light-gray},  % choose the background color. You must add \usepackage{color}
showspaces=false,               % show spaces adding particular underscores
showstringspaces=false,         % underline spaces within strings
showtabs=false,                 % show tabs within strings adding particular underscores
frame=single,           % adds a frame around the code
tabsize=2,          % sets default tabsize to 2 spaces
captionpos=b,           % sets the caption-position to bottom
breaklines=true,        % sets automatic line breaking
breakatwhitespace=false,    % sets if automatic breaks should only happen at whitespace
escapeinside={\%*}{*)}          % if you want to add a comment within your code
}


\title{ILA Modeling of Interrupt}
\author{}

\date{Draft Working Document: \today}

\begin{document}
\maketitle

\providecommand{\bd}[0]{\mathbb{B}}
\providecommand{\st}[1]{\mathrm{#1}}
\providecommand{\ft}[1]{\mathtt{#1}}

%%%%%%%%%%%%%%%%%%%%%%%%%%%%%%%%%%%%% Body %%%%%%%%%%%%%%%%%%%%%%%%%%%%%%%%%%%%

% Section on how to use ILA to model interrupts.

\section*{Interrupt}
For general purpose processors, the instruction-level abstraction (ILA) can not 
only capture the operations defined by its instruction-set architecture, but also
the behavior of interrupts.
As defined in the ILA definition, an instruction can be fetched from both the 
input port and the architectural states.

%
Take a processor with interrupt service routine (ISR) stored in a certain range 
of instruction memory, such as 8051 micro-controller, as an example.
When the interrupt signal is raised, the processor stores the current program 
counter into the stack and updates the program counter to the entry point of
the ISR.
This can be modeled with the following strategies:
%
\begin{enumerate}
\item Include the interrupt signal as part of the instruction.
\item Evaluate the decode functions of normal instructions to \textit{false} if 
    any interrupt signal is raised.
\item Model the interrupt handling with a new instruction whose decode 
    function is evaluated to \textit{true} when the interrupt signal is raised.
\item The next state function of the new interrupt instruction updates the 
    program counter to the ISR entry point.
\end{enumerate}

%
Note that, in this example, instructions in the ISR are also normal 
instructions. 
%
The below shows part of an ILA template of the above example in modeling 
interrupts.
\lstinputlisting{code/interruptExample.py}

%\bibliography{refs}
%\bibliographystyle{unsrt}

\end{document}

