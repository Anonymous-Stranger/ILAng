%%%%%%%%%%%%%%%%%%%%%%%%%%%%%%%%%%%%%%%%%%%%%%%%%%%%%%%%%%%%%%%%%%%%%%%%%%%%%%%%
%2345678901234567890123456789012345678901234567890123456789012345678901234567890
%        1         2         3         4         5         6         7         8

\documentclass[letterpaper, 11 pt]{article}  % Comment this line out
                                                          % if you need a4paper
%\documentclass[a4paper, 10pt, conference]{ieeeconf}      % Use this line for a4
                                                          % paper
\usepackage{url}
\usepackage{graphicx}
\usepackage{amsmath}
\usepackage{amssymb}
\usepackage{amsfonts}
\usepackage{proof}
\usepackage{tikz}
\usepackage[margin=1.2in]{geometry}


\usepackage{color}
\definecolor{light-gray}{gray}{0.95}
\usepackage{listings}
\lstset{ %
language=Python,                % choose the language of the code
basicstyle=\footnotesize,       % the size of the fonts that are used for the code
numbers=left,                   % where to put the line-numbers
numberstyle=\footnotesize,      % the size of the fonts that are used for the line-numbers
stepnumber=1,                   % the step between two line-numbers. If it is 1 each line will be numbered
numbersep=5pt,                  % how far the line-numbers are from the code
backgroundcolor=\color{light-gray},  % choose the background color. You must add \usepackage{color}
showspaces=false,               % show spaces adding particular underscores
showstringspaces=false,         % underline spaces within strings
showtabs=false,                 % show tabs within strings adding particular underscores
frame=single,           % adds a frame around the code
tabsize=2,          % sets default tabsize to 2 spaces
captionpos=b,           % sets the caption-position to bottom
breaklines=true,        % sets automatic line breaking
breakatwhitespace=false,    % sets if automatic breaks should only happen at whitespace
escapeinside={\%*}{*)}          % if you want to add a comment within your code
}



\title{ILA Modeling of Concurrency}
\author{}

\date{Draft Working Document: \today}

\begin{document}
\maketitle

\providecommand{\bd}[0]{\mathbb{B}}
\providecommand{\st}[1]{\mathrm{#1}}
\providecommand{\ft}[1]{\mathtt{#1}}

\section*{Concurrency}

In this scenario we consider two different design options: physical and logical 
multi-core (logical is simultaneous multithreading). 

\subsection*{Physical multi-core} 
In a physical multi-core design computational resources are duplicated. 
Modern CPUs contain 2-24 physical cores, while GPUs contain thousands 
of physical cores. Since our hierarchical-ILA supports concurrent 
execution of instructions, it can be used to model such multi-core 
designs. More precisely, an ILA for a multi-core design includes 
many identical child-ILAs. Note that while multi-core designs are based 
on the notion of duplicating computational resources, the way they 
operate is different. The most obvious example is CPU vs. GPU. In most 
cases, execution on different GPU cores is independent, unlike the case 
of a CPU. We need to decide if we add this modeling power to the ILA.

\subsection*{Logical multi-core, aka SMT, HyperThreading}
In this case, the computational resources are not duplicated. 
Instead, the architectural state (e.g. control and general 
purpose registers) are duplicated. Using this approach, if 
the CPU includes one core, the OS views it as two \emph{logical}
 cores. The management of the physical core 
 (the actual computational resource) is done at the 
 hardware level, i.e., by the CPU and can be captured by the ILA.

%\bibliography{refs}
%\bibliographystyle{unsrt}

\end{document}

