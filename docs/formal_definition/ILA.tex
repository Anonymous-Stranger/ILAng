%%%%%%%%%%%%%%%%%%%%%%%%%%%%%%%%%%%%%%%%%%%%%%%%%%%%%%%%%%%%%%%%%%%%%%%%%%%%%%%%
%2345678901234567890123456789012345678901234567890123456789012345678901234567890
%        1         2         3         4         5         6         7         8

\documentclass[letterpaper, 11 pt]{article}  % Comment this line out
                                                          % if you need a4paper
%\documentclass[a4paper, 10pt, conference]{ieeeconf}      % Use this line for a4
                                                          % paper
\usepackage{url}
\usepackage{graphicx}
\usepackage{amsmath}
\usepackage{amssymb}
\usepackage{amsfonts}
\usepackage{proof}
\usepackage{tikz}
\usepackage[margin=1.2in]{geometry}

\title{Formal Definition for Instruction-Level Abstraction}
\author{
    \begin{tabular}{ccccc}
    Bo-Yuan~Huang & Pramod~Subramanyan & Hongce~Zhang & Aarti~Gupta$^\dagger$ & Sharad~Malik \\
    \multicolumn{5}{c} {Departments of Computer Science$^\dagger$ and Electrical Engineering, Princeton University} \\
    \end{tabular}
}

\date{Draft Working Document: \today}

\begin{document}
\maketitle

%\providecommand{\bd}[0]{\mathbb{B}}
\providecommand{\st}[1]{\mathrm{#1}}
\providecommand{\ft}[1]{\mathtt{#1}}

\section*{Formal Definition for Instruction-Level Abstraction}

% Some preliminary background for ILA
Instruction-level abstraction (ILA) provides an interface between different 
components in heterogeneous systems.
It is an abstraction that formally models firmware-visible behaviors of a hardware
module, such as a programmable processor or application specific accelerator.
Similar to the instruction-set architecture (ISA) that defines the operation of
a processor, an ILA models the operation of a hardware module as a set of 
instructions, where each instruction represents a certain operation over the 
states.

% Start describing what is an ILA and the formal definition, with examples.
% Probably with some words about "what is an instruction".
% Remember, the definition needs to be backward compatable with processor ISA.
\subsection*{Instruction-Level Abstraction}
% State variable and input variables
ILA is an abstraction of the hardware and serves as the interface between 
the hardware and other components.
Therefore, only states that are visible and persistent across instructions need
to be modeled.
% State variables
We let $S$ be the vector of those state variables, where each element can be 
either a Boolean variable, a bitvector variable, or a memory variable.
In the ILA of a processor, $S$ would contain the architectural registers, flag
bits, data memory, and the instruction memory.
As for accelerators, $S$ could contain memory-mapped registers, internal buffers, 
output ports to on-chip interconnect, and so on.
% Input ports
In addition to state variables, an ILA also contains a vector of input variables
$W$ that models the input ports of the hardware module, such as the interrupt 
signals for processors and command inputs for accelerators.

% Execution model: Fetch/Decode/Execute
% Fetch and Valid
In an ILA, hardware operations are abstracted into a set of instructions 
following the fetch/decode/execute model as processors.
%
An ILA fetches the \textit{opcode} from its states $S$ and inputs $W$.
The opcode is a bitvector that will later be decoded to determine which 
instruction should be issued.
The fetch function indicating how is the opcode fetched is represented by 
$F : (S \times W) \mapsto bvec_w$ where $w$ is the width of the opcode.
Take programmable cores without interrupts for example, the opcode is fetched 
from the instruction memory pointed by the program counter, i.e. 
$F(S, W) \triangleq read( \st{IMEM}, \st{PC} )$.
In cases where interrupt matters, the fetch function could possibly be 
concatenating the interrupt signals with the value read from instruction memory.

% Valid function
Unlike programmable cores that always have an instruction to fetch, decode, and
execute, accelerators may be event-driven and only start executing instructions 
when a certain trigger occurs. 
We let $V : (S \times W) \mapsto \bd$ be the valid function that indicates when 
an ILA can execute.
%
For example, if an accelerator executes instructions only when the signals 
$\st{AddrInRange}$ and $\st{StateIdle}$ are asserted, then 
$V(S, W) \triangleq \st{AddrInRange} \wedge \st{StateIdle}$.
For programmable cores, there is always an instruction and the valid function 
is always true.

% Decode
Each instruction is associated with a decode function 
$\delta_i : bvec_w \mapsto \bd$ that takes the fetched opcode as input and 
tells whether the instruction is \textit{issued}.
Take 8051 microcontroller as an example, the opcode of the \textit{JZ} 
instruction is $\texttt{0x60}$, therefore 
$\delta_{JZ} (\st{opcode}) \triangleq \st{opcode} = \texttt{0x60}$.
The set of all decode functions in the ILA is denoted as 
$D = \{ \delta_i ~|~ 0 \leq i \leq C \}$, where $C$ is the number of 
instructions.

% Execution
Besides the decode function, every instruction is also associated with a 
next state function $N_i : (S \times W) \mapsto S$ to represent the state updates 
when an instruction is executed.
For example, the \textit{INC} instruction (opcode $\texttt{0x4}$) in the 8051 
microcontroller increments the accumulator, therefore the next state function 
of the accumulator is $N_4[\st{ACC}] = ACC + 1$. 
Note that the operation defined in $N_i$ is atomic and an instruction can be 
executed only if it has been issued.
The set of all next state functions in the ILA is denoted as 
$N = \{ N_i ~|~ 0 \leq i \leq C \}$, where $C$ is the number of instructions.

To summarize, ILA provides a uniform interface between hardware components 
in heterogeneous systems by formally abstracting firmware-visible behaviors of 
hardware into sets of instructions under the fetch/decode/execute model. 
An ILA is defined as follow:

\begin{eqnarray}
  \mathrm{A} &=& \langle S, W, F, V, D, N \rangle \text{, where} \nonumber \\
      &S& \text{ is the vector of state variables, } \nonumber \\
      &W& \text{ is the vector of input variables, } \nonumber \\
      &V& :(S\times W) \mapsto \bd \text{ is the valid function, } \nonumber \\
      &F& :(S\times W) \mapsto bvec_w \text{ is the fetch function,} \nonumber \\
      &D& = \{ \delta_i : bvec_w \mapsto \bd \} 
            \text{ is the set of decode functions, and} \nonumber \\
      &N& = \{ N_i : (S \times W) \mapsto S \} 
            \text{ is the set of next state functions.} \nonumber
\end{eqnarray}

% Talk about child-ILAs. What is child-ILA and why do we need them.
\subsection*{Hierarchical ILA}
% Why hierarchical ILA and example
While ILA serves as the interface between different components by 
abstracting hardware behaviors into sets of instructions, there are cases where
we would like to model the hardware in different levels of abstraction.
% Example of string copy
For example, in the specification of Intel x86 architecture, the string copy 
instruction \texttt{REP STOS} involves a sequence of bytes moves.
The instruction causes the associated \texttt{STOS} be repeated until the count
in register \texttt{ECX} is decremented to $0$ \cite{intel-manual}.
Such operation should not be represented as one single state update if we want to
model the behavior more precisely.
% Example of AES encryption
Another example is the \texttt{START\_ENCRYPT} instruction that starts the 
encryption operation of an AES accelerator, of which the implementation contains 
message loading, encryption computing, and output storing.

In hierarchical ILA, we allow an ILA to contain \textit{child-ILAs} for complex
instructions to help a manage different level of abstraction.
Similar to an ILA, a child-ILA follows the fetch/decode/execute model and 
contains a set of \textit{child-instructions} to help represent complex 
instructions.
Take \texttt{START\_ENCRYPT} for example, the operation can now be represented by 
a child-ILA containing one child-instruction for message loading, one for doing 
encryption, and another for storing the results.
The ILA containing the child-ILA is called the \textit{parent-ILA}.

% sub-instruction and micro-instruction
To distinguish between specified behavior and possible implementation, every 
child-ILA is associated with an auxiliary variable $u \in \bd$. 
$u = 1$ means that the behavior modeled by the child-instructions are required
in the specification.
We call such child-instructions \textit{sub-instructions}.
On the other hand, $u = 0$ means that the child-instructions are representing 
one possible implementation of the complex instruction, and we call them
\textit{micro-instructions}.

% formalism of child-ILA
A child-ILA is defined similarly as an ILA.
It has its own state variables, valid function, fetch function, decode functions, 
and next state functions denoted as $S^\mu$, $V^\mu$, $F^\mu$, $D^\mu$, and 
$N^\mu$ respectively.
% States
$S^\mu$ can contain shared states in $S$ of its parent ILA, and every update to
the shared states are synchronized.
% Valid
The valid function $V^\mu : S^\mu \mapsto \bd$ indicates when the child-ILA
can execute.
% Fetch
The fetch function $F^\mu : S^\mu \mapsto bvec_l$, where $l$ is the width of the
opcode, models how the opcode is fetched from the states.
% Decode
The decode function $D^\mu$ is defined exactly the same as an ILA to tell if 
any child-instruction is issued.
% Next
State update of each child-instruction is modeled by next state functions
$N^\mu_j : S^\mu \mapsto S^\mu$, and $N^\mu$ is the set of next state functions 
for all child-instructions.

\iffalse
Summarizing the above definition, a child-ILA is defined as follow:
\begin{eqnarray}
  A^\mu &=& \langle S^\mu, V^\mu, F^\mu, D^\mu, N^\mu \rangle 
        \text{, where} \nonumber \\
    &S^\mu& \text{ is the vector of state variables, } \nonumber \\
    &V^\mu& : S^\mu \mapsto \bd \text{ is the valid function, } \nonumber \\
    &F^\mu& : S^\mu \mapsto bvec_l \text{ is the fetch function, } \nonumber \\
    &D^\mu& = \{ \delta_j^\mu : bvec_l \mapsto \bd \}
            \text{ is the decode function, and} \nonumber \\
    &N^\mu& = \{ N_j^\mu : S^\mu \mapsto S^\mu \}
            \text{ is the next state function.} \nonumber
\end{eqnarray}
\fi

% Summarize ILA
With the extension of hierarchical ILA, ILA can now model the hardware at 
different levels of abstraction.
%
A hierarchical ILA is defined as
$A = \langle S, W, V, F, D, N, L \rangle$, where 
$L = \{ (u^{\mu_p}, A^{\mu_p}), \dots \}$ is the set of child-ILAs.
$u^{\mu_p}$ is the auxiliary variable showing if $A^{\mu_p}$ is composed of 
sub-instructions or micro-instructions.
%
Similarly, child-ILAs can also contain other child-ILAs and are defined exactly 
the same, except that micro-ILAs can only contain micro-ILAs.

% Talk about the instruction sequencing, implicitly determined in state updates.
% Also the program structure such as the loop and entry/exit.
\subsubsection*{Micro-program and sub-program}
% The sequencing of the micro-program and example such as micro code and loops
Once a child-ILA starts executing, it then steps through a sequence of 
child-instructions where the sequencing is implicitly determined by the state
update itself.
This can be understood as a \textit{child-program} even though the control 
location is not defined on a specific variable as usual.
%
Take the string copy instruction for example, the child-program representing 
the instruction contains a loop whose loop body is the child-instruction that 
moves a byte at a time.
% 
We call a child-program \textit{sub-program} if it is composed of 
sub-instructions, and call it \textit{micro-program} if it is composed of 
micro-instructions. 

%The ILA and its child-ILAs execute asynchronously and communicate through 
%shared state variables. 
A child-ILA $A^\mu$ starts executing when the valid function $V^\mu$ changes 
from $0$ to $1$, indicating that there is one child-instruction being issued.
%
This could happen when a certain instruction in its parent-ILA, the parent 
instruction, is executed and modifies the shared states so that $V^\mu$ becomes 
true.
For example, the \texttt{START\_ENCRYPT} instruction of the AES accelerator 
updates the shared state \textit{aes\_state}, which then starts the child-ILA
with the child-instruction for message loading being issued.
%
It is also possible that $V^\mu$ becomes true at the same time as the parent
instruction is issued.
Take string copy for example, the child-instruction for the first
byte move is issued at the same time as the string copy instruction is issued.
In such cases, the execution of the parent instruction is trivial.

% Describe how is the state update defined and the relation between ILAs.
\subsection*{Execution Semantics}
The ILA and its child-ILAs execute asynchronously and communicate through shared
state variables. 
Updates of the shared states are synchronized.
When an instruction is executed, the shared states in child-ILAs will also be 
updated; when a child-instructions is executed, the shared states in the parent
ILA will also be updated.

The execution of an instruction in the ILA is defined by 
rule~(\ref{rule:ILA_exec}).
%
\begin{equation}
  \infer{S \rightsquigarrow S'}
        {V(S, W) & \delta_i(F(S, W)) & S' = N_i(S, W)}
  \label{rule:ILA_exec}
\end{equation}
%
The rule says that an ILA can transition from state $S$ to $S'$ if the following 
conditions are satisfied:
\begin{enumerate}
  \item An instruction is being triggered: $V(S, W) = 1$.
  \item The $i$-th instruction type is issued: $\delta_i(F(S, W)) = 1$.
  \item State update of each variable in the vector $S'$ is as defined by 
        $N_i (S, W)$.
\end{enumerate}

The execution of a child-instruction in child-ILA is defined by 
rule~(\ref{rule:cILA_exec}).
\begin{equation}
  \infer{S^\mu \rightsquigarrow S'^\mu}
        {V^\mu(S^\mu) & \delta_i^\mu(F^\mu(S^\mu)) & S'^\mu = N_i^\mu(S^\mu)}
  \label{rule:cILA_exec}
\end{equation}
The rule says that a child-ILA can transition from state $S^\mu$ to $S'^\mu$ if 
the following conditions are satisfied:
\begin{enumerate}
  \item The child-ILA is activate: $V(S^\mu) = 1$.
  \item The $j$-th child-instruction is issued: $\delta_j^\mu(S^\mu) = 1$.
  \item State update of each variable in the vector $S'^\mu$ is as defined by 
        $N_j^\mu (S^\mu)$.
\end{enumerate}

There are two implications of these two rules.
First, each state update defined in $N_i$ or $N_j^\mu$ is atomic.
Second, the decode function only indicates whether an instruction is issued, it 
does not say anything about execution.

The synchronization of the shared states is defined by 
rules~(\ref{rule:ILA_sync}) and ~(\ref{rule:cILA_sync}).

\begin{equation}
  \infer{S^\mu \overset{S'}{\rightsquigarrow} S'^\mu}
        {S \rightsquigarrow S'}
  \label{rule:ILA_sync}
\end{equation}

\begin{equation}
  \infer{S \overset{S'^\mu}{\rightsquigarrow} S'}
        {S^\mu \rightsquigarrow S'^\mu}
  \label{rule:cILA_sync}
\end{equation}

The notation $S \overset{S'^\mu}{\rightsquigarrow} S'$ means that states in the 
ILA change from $S$ to $S'$ with all the shared states between $S$ and $S^\mu$
updates to the corresponding value in $S'^\mu$ while other states in $S$ remain 
unchanged.
The notation $S^\mu \overset{S'}{\rightsquigarrow} S'^\mu$ is defined in a 
similar way.

% Use the above to talk about the importance and advantage of modeling 
% concurrency between instructions and interleavings.
% Illustrate with various kinds of cases.
\subsection*{Instruction Concurrency}
% Why this is important and needed
% Discuss with the above semantics
% Illustrate various cases of concurrency

% Concurrency between instructions/accelerators with examples
%ILA serves as the interface between the software and the hardware by abstracting
%firmware-visible behavior into sets of instructions. 

ILA not only serves as the interface between hardware modules but 
also allows us to model the concurrency behavior across instructions. 
Modeling such concurrency is important and necessary in verifying heterogeneous 
systems due to the parallel interaction between hardware components.
%
For example, consider a scenario where a firmware utilizes the AES accelerator
to encrypt a block of message. 
It first uses the string copy instruction to move the message into a certain
range of memory; then uses memory-mapped I/O (MMIO) to trigger the encryption
operation, which depends on the message; finally, it reads back the result after
checking the status of the accelerator.
%
Here both the string copy and the encryption are non-blocking instructions. 
They allow other instructions to be issued and executed even if they are not
finished yet.
Precisely modeling the concurrent interaction between non-blocking instructions 
is necessary for correct verification.

% Our ability to model the concurrency with the ILA, explain with the execution
% semantics.
With the help of hierarchical ILA, non-blocking operations of complex 
instructions can be represented by child-instructions.
The execution semantics of the ILA can model the scenario which issues a new
instruction even when other child-ILAs are still executing.
%
Note that the signals on input ports, which are usually used to trigger 
instructions, are not required to remain unchanged during the execution of 
child-ILAs.
%The execution semantics of the ILA let us model the concurrent behavior that 
%issues a new instruction while the child-ILAs of previous instructions are
%still executing.
The state updates of these instructions and child-instructions are atomic, as 
defined by the next state functions, and can be interleaved with others.
%
Therefore we can then utilize the works on instruction interleaving to model
and verify the concurrency interaction.

% We can also model the case where commit follows program order.
Note that ILA is not restricted to representing non-blocking operations of 
complex instructions. 
Other cases can have corresponding abstractions with different decode 
functions and next state functions depending on the underlying hardware behavior.
For example, a simple processor, where new instructions cannot be issued until
previous instructions are complete, can be modeled by ensuring the program
counter be updated at the last child-instruction.


\providecommand{\bd}[0]{\mathbb{B}}
\providecommand{\st}[1]{\mathrm{#1}}
\providecommand{\ft}[1]{\mathtt{#1}}

%\section*{Formal Definition for Instruction-Level Abstraction}

% Some preliminary background for ILA
Instruction-level abstraction (ILA) provides an interface between different 
components in heterogeneous systems.
It is an abstraction that formally models firmware-visible behaviors of a hardware
module, such as a programmable processor or application specific accelerator.
Similar to the instruction-set architecture (ISA) that defines the operation of
a processor, an ILA models the operation of a hardware module as a set of 
instructions, where each instruction represents a certain operation over the 
states.

% Start describing what is an ILA and the formal definition, with examples.
% Probably with some words about "what is an instruction".
% Remember, the definition needs to be backward compatable with processor ISA.
\subsection*{Instruction-Level Abstraction}
% State variable and input variables
ILA is an abstraction of the hardware and serves as the interface between 
the hardware and other components.
Therefore, only states that are visible and persistent across instructions need
to be modeled.
% State variables
We let $S$ be the vector of those state variables, where each element can be 
either a Boolean variable, a bitvector variable, or a memory variable.
In the ILA of a processor, $S$ would contain the architectural registers, flag
bits, data memory, and the instruction memory.
As for accelerators, $S$ could contain memory-mapped registers, internal buffers, 
output ports to on-chip interconnect, and so on.
% Input ports
In addition to state variables, an ILA also contains a vector of input variables
$W$ that models the input ports of the hardware module, such as the interrupt 
signals for processors and command inputs for accelerators.

% Execution model: Fetch/Decode/Execute
% Fetch and Valid
In an ILA, hardware operations are abstracted into a set of instructions 
following the fetch/decode/execute model as processors.
%
An ILA fetches the \textit{opcode} from its states $S$ and inputs $W$.
The opcode is a bitvector that will later be decoded to determine which 
instruction should be issued.
The fetch function indicating how is the opcode fetched is represented by 
$F : (S \times W) \mapsto bvec_w$ where $w$ is the width of the opcode.
Take programmable cores without interrupts for example, the opcode is fetched 
from the instruction memory pointed by the program counter, i.e. 
$F(S, W) \triangleq read( \st{IMEM}, \st{PC} )$.
In cases where interrupt matters, the fetch function could possibly be 
concatenating the interrupt signals with the value read from instruction memory.

% Valid function
Unlike programmable cores that always have an instruction to fetch, decode, and
execute, accelerators may be event-driven and only start executing instructions 
when a certain trigger occurs. 
We let $V : (S \times W) \mapsto \bd$ be the valid function that indicates when 
an ILA can execute.
%
For example, if an accelerator executes instructions only when the signals 
$\st{AddrInRange}$ and $\st{StateIdle}$ are asserted, then 
$V(S, W) \triangleq \st{AddrInRange} \wedge \st{StateIdle}$.
For programmable cores, there is always an instruction and the valid function 
is always true.

% Decode
Each instruction is associated with a decode function 
$\delta_i : bvec_w \mapsto \bd$ that takes the fetched opcode as input and 
tells whether the instruction is \textit{issued}.
Take 8051 microcontroller as an example, the opcode of the \textit{JZ} 
instruction is $\texttt{0x60}$, therefore 
$\delta_{JZ} (\st{opcode}) \triangleq \st{opcode} = \texttt{0x60}$.
The set of all decode functions in the ILA is denoted as 
$D = \{ \delta_i ~|~ 0 \leq i \leq C \}$, where $C$ is the number of 
instructions.

% Execution
Besides the decode function, every instruction is also associated with a 
next state function $N_i : (S \times W) \mapsto S$ to represent the state updates 
when an instruction is executed.
For example, the \textit{INC} instruction (opcode $\texttt{0x4}$) in the 8051 
microcontroller increments the accumulator, therefore the next state function 
of the accumulator is $N_4[\st{ACC}] = ACC + 1$. 
Note that the operation defined in $N_i$ is atomic and an instruction can be 
executed only if it has been issued.
The set of all next state functions in the ILA is denoted as 
$N = \{ N_i ~|~ 0 \leq i \leq C \}$, where $C$ is the number of instructions.

To summarize, ILA provides a uniform interface between hardware components 
in heterogeneous systems by formally abstracting firmware-visible behaviors of 
hardware into sets of instructions under the fetch/decode/execute model. 
An ILA is defined as follow:

\begin{eqnarray}
  \mathrm{A} &=& \langle S, W, F, V, D, N \rangle \text{, where} \nonumber \\
      &S& \text{ is the vector of state variables, } \nonumber \\
      &W& \text{ is the vector of input variables, } \nonumber \\
      &V& :(S\times W) \mapsto \bd \text{ is the valid function, } \nonumber \\
      &F& :(S\times W) \mapsto bvec_w \text{ is the fetch function,} \nonumber \\
      &D& = \{ \delta_i : bvec_w \mapsto \bd \} 
            \text{ is the set of decode functions, and} \nonumber \\
      &N& = \{ N_i : (S \times W) \mapsto S \} 
            \text{ is the set of next state functions.} \nonumber
\end{eqnarray}

% Talk about child-ILAs. What is child-ILA and why do we need them.
\subsection*{Hierarchical ILA}
% Why hierarchical ILA and example
While ILA serves as the interface between different components by 
abstracting hardware behaviors into sets of instructions, there are cases where
we would like to model the hardware in different levels of abstraction.
% Example of string copy
For example, in the specification of Intel x86 architecture, the string copy 
instruction \texttt{REP STOS} involves a sequence of bytes moves.
The instruction causes the associated \texttt{STOS} be repeated until the count
in register \texttt{ECX} is decremented to $0$ \cite{intel-manual}.
Such operation should not be represented as one single state update if we want to
model the behavior more precisely.
% Example of AES encryption
Another example is the \texttt{START\_ENCRYPT} instruction that starts the 
encryption operation of an AES accelerator, of which the implementation contains 
message loading, encryption computing, and output storing.

In hierarchical ILA, we allow an ILA to contain \textit{child-ILAs} for complex
instructions to help a manage different level of abstraction.
Similar to an ILA, a child-ILA follows the fetch/decode/execute model and 
contains a set of \textit{child-instructions} to help represent complex 
instructions.
Take \texttt{START\_ENCRYPT} for example, the operation can now be represented by 
a child-ILA containing one child-instruction for message loading, one for doing 
encryption, and another for storing the results.
The ILA containing the child-ILA is called the \textit{parent-ILA}.

% sub-instruction and micro-instruction
To distinguish between specified behavior and possible implementation, every 
child-ILA is associated with an auxiliary variable $u \in \bd$. 
$u = 1$ means that the behavior modeled by the child-instructions are required
in the specification.
We call such child-instructions \textit{sub-instructions}.
On the other hand, $u = 0$ means that the child-instructions are representing 
one possible implementation of the complex instruction, and we call them
\textit{micro-instructions}.

% formalism of child-ILA
A child-ILA is defined similarly as an ILA.
It has its own state variables, valid function, fetch function, decode functions, 
and next state functions denoted as $S^\mu$, $V^\mu$, $F^\mu$, $D^\mu$, and 
$N^\mu$ respectively.
% States
$S^\mu$ can contain shared states in $S$ of its parent ILA, and every update to
the shared states are synchronized.
% Valid
The valid function $V^\mu : S^\mu \mapsto \bd$ indicates when the child-ILA
can execute.
% Fetch
The fetch function $F^\mu : S^\mu \mapsto bvec_l$, where $l$ is the width of the
opcode, models how the opcode is fetched from the states.
% Decode
The decode function $D^\mu$ is defined exactly the same as an ILA to tell if 
any child-instruction is issued.
% Next
State update of each child-instruction is modeled by next state functions
$N^\mu_j : S^\mu \mapsto S^\mu$, and $N^\mu$ is the set of next state functions 
for all child-instructions.

\iffalse
Summarizing the above definition, a child-ILA is defined as follow:
\begin{eqnarray}
  A^\mu &=& \langle S^\mu, V^\mu, F^\mu, D^\mu, N^\mu \rangle 
        \text{, where} \nonumber \\
    &S^\mu& \text{ is the vector of state variables, } \nonumber \\
    &V^\mu& : S^\mu \mapsto \bd \text{ is the valid function, } \nonumber \\
    &F^\mu& : S^\mu \mapsto bvec_l \text{ is the fetch function, } \nonumber \\
    &D^\mu& = \{ \delta_j^\mu : bvec_l \mapsto \bd \}
            \text{ is the decode function, and} \nonumber \\
    &N^\mu& = \{ N_j^\mu : S^\mu \mapsto S^\mu \}
            \text{ is the next state function.} \nonumber
\end{eqnarray}
\fi

% Summarize ILA
With the extension of hierarchical ILA, ILA can now model the hardware at 
different levels of abstraction.
%
A hierarchical ILA is defined as
$A = \langle S, W, V, F, D, N, L \rangle$, where 
$L = \{ (u^{\mu_p}, A^{\mu_p}), \dots \}$ is the set of child-ILAs.
$u^{\mu_p}$ is the auxiliary variable showing if $A^{\mu_p}$ is composed of 
sub-instructions or micro-instructions.
%
Similarly, child-ILAs can also contain other child-ILAs and are defined exactly 
the same, except that micro-ILAs can only contain micro-ILAs.

% Talk about the instruction sequencing, implicitly determined in state updates.
% Also the program structure such as the loop and entry/exit.
\subsubsection*{Micro-program and sub-program}
% The sequencing of the micro-program and example such as micro code and loops
Once a child-ILA starts executing, it then steps through a sequence of 
child-instructions where the sequencing is implicitly determined by the state
update itself.
This can be understood as a \textit{child-program} even though the control 
location is not defined on a specific variable as usual.
%
Take the string copy instruction for example, the child-program representing 
the instruction contains a loop whose loop body is the child-instruction that 
moves a byte at a time.
% 
We call a child-program \textit{sub-program} if it is composed of 
sub-instructions, and call it \textit{micro-program} if it is composed of 
micro-instructions. 

%The ILA and its child-ILAs execute asynchronously and communicate through 
%shared state variables. 
A child-ILA $A^\mu$ starts executing when the valid function $V^\mu$ changes 
from $0$ to $1$, indicating that there is one child-instruction being issued.
%
This could happen when a certain instruction in its parent-ILA, the parent 
instruction, is executed and modifies the shared states so that $V^\mu$ becomes 
true.
For example, the \texttt{START\_ENCRYPT} instruction of the AES accelerator 
updates the shared state \textit{aes\_state}, which then starts the child-ILA
with the child-instruction for message loading being issued.
%
It is also possible that $V^\mu$ becomes true at the same time as the parent
instruction is issued.
Take string copy for example, the child-instruction for the first
byte move is issued at the same time as the string copy instruction is issued.
In such cases, the execution of the parent instruction is trivial.

% Describe how is the state update defined and the relation between ILAs.
\subsection*{Execution Semantics}
The ILA and its child-ILAs execute asynchronously and communicate through shared
state variables. 
Updates of the shared states are synchronized.
When an instruction is executed, the shared states in child-ILAs will also be 
updated; when a child-instructions is executed, the shared states in the parent
ILA will also be updated.

The execution of an instruction in the ILA is defined by 
rule~(\ref{rule:ILA_exec}).
%
\begin{equation}
  \infer{S \rightsquigarrow S'}
        {V(S, W) & \delta_i(F(S, W)) & S' = N_i(S, W)}
  \label{rule:ILA_exec}
\end{equation}
%
The rule says that an ILA can transition from state $S$ to $S'$ if the following 
conditions are satisfied:
\begin{enumerate}
  \item An instruction is being triggered: $V(S, W) = 1$.
  \item The $i$-th instruction type is issued: $\delta_i(F(S, W)) = 1$.
  \item State update of each variable in the vector $S'$ is as defined by 
        $N_i (S, W)$.
\end{enumerate}

The execution of a child-instruction in child-ILA is defined by 
rule~(\ref{rule:cILA_exec}).
\begin{equation}
  \infer{S^\mu \rightsquigarrow S'^\mu}
        {V^\mu(S^\mu) & \delta_i^\mu(F^\mu(S^\mu)) & S'^\mu = N_i^\mu(S^\mu)}
  \label{rule:cILA_exec}
\end{equation}
The rule says that a child-ILA can transition from state $S^\mu$ to $S'^\mu$ if 
the following conditions are satisfied:
\begin{enumerate}
  \item The child-ILA is activate: $V(S^\mu) = 1$.
  \item The $j$-th child-instruction is issued: $\delta_j^\mu(S^\mu) = 1$.
  \item State update of each variable in the vector $S'^\mu$ is as defined by 
        $N_j^\mu (S^\mu)$.
\end{enumerate}

There are two implications of these two rules.
First, each state update defined in $N_i$ or $N_j^\mu$ is atomic.
Second, the decode function only indicates whether an instruction is issued, it 
does not say anything about execution.

The synchronization of the shared states is defined by 
rules~(\ref{rule:ILA_sync}) and ~(\ref{rule:cILA_sync}).

\begin{equation}
  \infer{S^\mu \overset{S'}{\rightsquigarrow} S'^\mu}
        {S \rightsquigarrow S'}
  \label{rule:ILA_sync}
\end{equation}

\begin{equation}
  \infer{S \overset{S'^\mu}{\rightsquigarrow} S'}
        {S^\mu \rightsquigarrow S'^\mu}
  \label{rule:cILA_sync}
\end{equation}

The notation $S \overset{S'^\mu}{\rightsquigarrow} S'$ means that states in the 
ILA change from $S$ to $S'$ with all the shared states between $S$ and $S^\mu$
updates to the corresponding value in $S'^\mu$ while other states in $S$ remain 
unchanged.
The notation $S^\mu \overset{S'}{\rightsquigarrow} S'^\mu$ is defined in a 
similar way.

% Use the above to talk about the importance and advantage of modeling 
% concurrency between instructions and interleavings.
% Illustrate with various kinds of cases.
\subsection*{Instruction Concurrency}
% Why this is important and needed
% Discuss with the above semantics
% Illustrate various cases of concurrency

% Concurrency between instructions/accelerators with examples
%ILA serves as the interface between the software and the hardware by abstracting
%firmware-visible behavior into sets of instructions. 

ILA not only serves as the interface between hardware modules but 
also allows us to model the concurrency behavior across instructions. 
Modeling such concurrency is important and necessary in verifying heterogeneous 
systems due to the parallel interaction between hardware components.
%
For example, consider a scenario where a firmware utilizes the AES accelerator
to encrypt a block of message. 
It first uses the string copy instruction to move the message into a certain
range of memory; then uses memory-mapped I/O (MMIO) to trigger the encryption
operation, which depends on the message; finally, it reads back the result after
checking the status of the accelerator.
%
Here both the string copy and the encryption are non-blocking instructions. 
They allow other instructions to be issued and executed even if they are not
finished yet.
Precisely modeling the concurrent interaction between non-blocking instructions 
is necessary for correct verification.

% Our ability to model the concurrency with the ILA, explain with the execution
% semantics.
With the help of hierarchical ILA, non-blocking operations of complex 
instructions can be represented by child-instructions.
The execution semantics of the ILA can model the scenario which issues a new
instruction even when other child-ILAs are still executing.
%
Note that the signals on input ports, which are usually used to trigger 
instructions, are not required to remain unchanged during the execution of 
child-ILAs.
%The execution semantics of the ILA let us model the concurrent behavior that 
%issues a new instruction while the child-ILAs of previous instructions are
%still executing.
The state updates of these instructions and child-instructions are atomic, as 
defined by the next state functions, and can be interleaved with others.
%
Therefore we can then utilize the works on instruction interleaving to model
and verify the concurrency interaction.

% We can also model the case where commit follows program order.
Note that ILA is not restricted to representing non-blocking operations of 
complex instructions. 
Other cases can have corresponding abstractions with different decode 
functions and next state functions depending on the underlying hardware behavior.
For example, a simple processor, where new instructions cannot be issued until
previous instructions are complete, can be modeled by ensuring the program
counter be updated at the last child-instruction.

\bibliography{refs}
\bibliographystyle{unsrt}

\end{document}

